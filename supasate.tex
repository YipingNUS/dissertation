\documentclass[a4paper,10pt,engthesis,doctor]{chula}

%% These are some useful LaTeX package
%% the list is taken from the bare_jnrl.tex of IEEEtran, by Michael Shell
%\usepackage[pdftex]{graphicx}           % for including graphics (eps and pdf)
%% NOTE: for dual use with latex and pdflatex, instead load graphicx like:
%\ifx\pdfoutput\undefined
%  \usepackage[dvips]{graphicx}
%  \graphicspath{{./figure/}}
%\else
%  \usepackage[pdftex]{graphicx}
%  \graphicspath{{./figure/}}
%\fi
\graphicspath{{./figure/}}

\usepackage{amssymb}
\usepackage[normalem]{ulem}
\usepackage[cmex10]{amsmath}    % math package provides several command and symbols
                                % cmex10 option force only type 1 font to be used
\interdisplaylinepenalty=2500   % amsmath set this value to 10000, if you want to adjust
                                % this, do so after use the package
\usepackage{algorithm}          % package for writing a pseudo code
\usepackage{algorithmic}        % package for writing a pseudo code
\usepackage{array}              % improvement to the tabular and array
%\usepackage{mdwmath}           % other useful math package for formating equations
%\usepackage{mdwtab}
\usepackage{subfig}             % for placing subfigures in one figures environment
                                % (for example, Fig 1a and 1b)
\usepackage{url}                % for breaking urls (use \url{http://www.example.com}
\usepackage{setspace}           % for double spacing
%\usepackage{times}             % for times roman font
%\usepackage{lscape}			   % for landscape page

\usepackage{multirow}
\usepackage{indentfirst}
\usepackage{natbib}

%% Environment, theorem, etc.
\newtheorem{thm}{Theorem}[chapter]
\newtheorem{cor}[thm]{Corollary}
\newtheorem{lem}[thm]{Lemma}
\newtheorem{prop}[thm]{Proposition}
\newtheorem{defn}[thm]{Definition}
\newtheorem{conj}[thm]{Conjecture}

% we use \mathbb{R} instead of \Re
\renewcommand{\Re}{\mathbb{R}}

\newcommand{\pspan}{\mbox{\sc span}^+}
\newcommand{\nspan}{\mbox{\sc span}^-}
\newcommand{\itr}{\mbox{\sc int}}
\newcommand{\co}{\mbox{\sc co}}
\newcommand{\ri}{\mbox{\sc ri}}
\newcommand{\ints}{\Omega}
\newcommand{\rspace}{\Re^3}
\newcommand{\ha}{\mathcal{H}}
\newcommand{\FindInts}{\mbox{\sc FindInts}}
\newcommand{\FindFG}{\mbox{\sc FindFG}}
\newcommand{\La}{\mathcal{U}}
\newcommand{\Lb}{{}^S\mathcal{U}}
\newcommand{\Lc}{{}^\mathfrak{F}\mathcal{U}}
\newcommand{\Ta}{\mathcal{P}}

%% For proof
\def\QED{\mbox{\rule[0pt]{1.3ex}{1.3ex}}} % for a filled box
\def\proof{\noindent\hspace{2em}{\itshape Proof: }}
\def\endproof{\hspace*{\fill}~\QED\par\endtrivlist\unskip\paragraph{}}

%% Correct bad hyphenation here
\hyphenation{net-works physico-mi-metics kij-sirikul cha-lerm-ek inta-nagon-wiwat tiny-os}

%%------------------------------------------------------------
%%-               SETTING THESIS PARAMETERS                  -
%%------------------------------------------------------------
\thesistitle                            % set the thesis tile (use {\wbr} for word break)
{ขั้นตอนวิธีการไม่ประสานเวลา{\wbr}เลียนแบบ{\wbr}หลักการทางฟิสิกส์{\wbr}โดย{\wbr}ปราศ{\wbr}จากความรู้ทาง{\wbr}เวลาแบบครอบคลุมสำหรับเครือข่ายตัวรับรู้ไร้สาย} % Thai Title
{LIGHTLY-SUPERVISED LEARNING METHODS FOR ONE-CLASS TEXT CLASSIFICATION}      % English title (auto upper case)

\authortitle                            % set the title of the author, e.g., Mr., Miss, Dr. etc.
{นาย}                                   % Thai Title of the author
{Mr.}                                   % English Title of the author
\thesisauthor                           % set the author name (must not include title (Mr., Miss, Dr., etc.)
{Yiping Jin}                          % Author name in Thai
{Yiping Jin}                  % Author name in English

\advisor                                               % set the advisor
{ผู้ช่วยศาสตราจารย์ ดร. เฉลิมเอก อินทนากรวิวัฒน์}               % Advisor name in Thai
{ผศ. ดร. เฉลิมเอก อินทนากรวิวัฒน์}                          % Advisor name in Thai with abbrev. title
{ASST. PROF. DITTAYA WANVARIE, Ph.D.}  % Advisor name in English with abbrev. title 
                                                       % (Capital letters except Ph.D.))

%% If the thesis has co-advisor, use this optional command
%\coadvisor
%{ผู้ช่วยศาสตราจารย์ ดร. เฉลิมเอก อินทนากรวิวัฒน์}  % Co-Advisor name in Thai
%{ผศ. ดร. เฉลิมเอก อินทนากรวิวัฒน์}  % Co-Advisor name in Thai with abbrev. title
%{Assistant Professor Chalermek Intanagonwiwat, Ph.D.}  % Co-Advisor name in English
%{ASST. PROF. CHALERMEK INTANAGONWIWAT, Ph.D.}          % Co-Advisor name in English with abbrev. title 
                                                       % (Capital letters except Ph.D.))

\faculty                                % set the faculty
{วิศวกรรมศาสตร์}                          % name of the faculty in Thai
{Science}                           % name of the faculty in English
\department                             % set the department
{วิศวกรรมคอมพิวเตอร์}                      % name of the department in Thai
{Mathematics and Computer Science}                  % name of the department in English
\fieldofstudy                           % set the field of study (าขิช)
{วิศวกรรมคอมพิวเตอร์}                      % Field in Thai
{Computer Engineering}                  % Field in English
\degree                                 % set the degree name
{วิศวกรรมศาสตรดุษฎีบัณฑิต}                   % Degree name in Thai
{Computer Science and Information Technology}                  % Degree name in English
\academicyear{2561}                     % Academic Year (in Thai Calendar)
\authorid{5972634023}                   % ID of the author
\keywords{Natural Language Processing, Text Classification, Semi-supervised Learning Methods, One-class Classification, Generalized Expectation Criteria, Na\"ive Bayes Classifier
}                                       % Keywords, for English abstract
\deanname                               % name of the dean
{รองศาสตราจารย์ ดร. บุญสม เลิศหิรัญวงศ์}                       % in Thai
{Associate Professor Boonsom Lerdhirunwong, Dr.Ing.}     % in English
%{ศาสตราจารย์ ดร. ดิเรก ลาวัณย์ศิริ}                           % in Thai
%{Professor Direk Lavansiri, Ph.D.}                      % in English

\committee                              % list of committee, please use \CommitteeBlock,\CommiteeBlockAdvisor,\CommitteBlockCoAdvisor
{
\CommitteeBlock{Chairman}{Professor Prabhas Chongstitvatana, Ph.D.}          % chairman
\CommitteeBlockAdvisor                  % pre-defined value for advisor
\CommitteeBlockCoAdvisor                % pre-defined value for co-advisor
\CommitteeBlock{Examiner}{Professor Boonserm Kijsirikul, Ph.D.}                % examiner
\CommitteeBlock{Examiner}{Assistant Professor Kultida Rojviboonchai, Ph.D.}     % examiner
\CommitteeBlock{External Examiner}{Associate Professor Anan Phonphoem, Ph.D.}  % examiner
}

%% Next is the example of Thai Committee
%% the author should use only one committee
%\committee                              % list of committee
%{
%\CommitteeBlock{ประธาน}{ศาสตราจารย์ ดร. ประภาส จงสถิตย์วัฒนา}   % chairman
%\CommitteeBlockAdvisor                  % pre-defined value for advisor
%\CommitteeBlockCoAdvisor                % pre-defined value for co-advisor
%\CommitteeBlock{กรรมการ}{ศาสตราจารย์ ดร. บุญเสริม กิจศิริกุล}       % examiner
%\CommitteeBlock{กรรมการ}{ผู้ช่วยศาสตราจารย์ ดร. กุลธิดา โรจน์วิบูลขัย} % examiner
%\CommitteeBlock{กรรมการภายนอก}{รองศาสตราจารย์ ดร. อนันต์ ผลเพิ่ม} % examiner
%}

\include{macros}
%%-------------------------------------------------------------------------------
%%-                               DOCUMENT                                      -
%%-------------------------------------------------------------------------------
\begin{document}
%% Thesis starts with Thai Cover
\makethaicover
\newpage

%% Followed by English Cover
\makeenglishcover
\newpage

%% Generate committee page
\makecommittee
\newpage
%% Thai Abstract Section
\include{abstractthai}
\newpage

%% English Abstract Section
\include{abstractenglish}
\newpage

% Acknowledgement Section
\begin{acknowledgements}
First and foremost, I would like to express my sincere gratitude to my advisor Asst. Prof. Dr. Dittaya Wanvarie, for her guidance, suggestions, and support.

I appreciate Prof. Dr. Prabhas Chongstitvatana, Prof. Dr. Chidchanok Lursinsap, Asst. Prof. Dr. Kritsada Sriphaew, for being my thesis committee and for their useful comments to improve my work. 
 
I greatly appreciate my reporting officer, Phu Le, head of R\&D Department at Knorex, who offerd me full support to pursue M.S. study at Chulalongkorn University. I learned a lot from him not only technical experience, but more importantly leadership and inter-personal skills.

 I am grateful to my parents for their endless support and understanding. The decision to move to Thailand and to pursue a Masters' study has not been easy. But my parents are always by my side and respect my decision.

Last but not least, I would like to thank Department of Mathematics and Computer Science for the excellent program. During my time at this department, I enjoyed the classes and interaction with professors and fellow students. 

\end{acknowledgements}

\newpage

% Table of Content
\tableofcontents        % generate table of content
\newpage
\listoftables           % generate list of tables
\newpage
\listoffigures          % generate list of figures
		
%% Main Content
\include{chapter1} 
\include{chapter2}
\include{chapter3}
\include{chapter4}
\include{chapter5}
\include{chapter6}
\include{chapter7}

%% References Section (must be placed before appendices)
\ULforem %%%
\setlength{\bibhang}{1.5cm}
\bibliographystyle{chulanat}
\bibliography{supasate} % speficy your bibtex file here (this example is supasate_dissertation.bib).

\normalem

%% Appendix Section
\numappendices{2}        % the number of appendices
\startappendix
\chapter{Publication} 
\label{app:math}

During my M.S study, I have published one papers as follows.

{\flushleft\bfseries International Conference Publication}
\begin{enumerate}
\item Yiping Jin, Dittaya Wanvarie., Phu Le. ``Combining Lightly-Supervised Text Classification Models for Accurate Contextual Advertising''. Proceedings of the Eighth International Joint Conference on Natural Language Processing (Volume 1: Long Papers). (2017). pp. 545--554.
\end{enumerate}

%\include{appendix2}

%% Biography Section
\begin{biography}
Yiping Jin was born in Beijing, China, on March, 1990.
He graduated from Beijing No. 4 High School, one of the most prestigious high schools in China, in 2008. 
Then, he received B.Comp. in Computer Science, from National University of Singapore, Singapore, in 2013 with First Class Honours Degree.
His bachelor degree has been supervised by Assoc. Prof. Dr. Min-Yen Kan.
He has received Singapore Ministry of Education and German Academic Exchange Service (DAAD) scholarship.
He currently works as Senior Research Scientist at Knorex Pte. Ltd. (Singapore) and serve as the lead of natural language processing and data science teams.
His field of interest includes various topics in artificial intellegence, such as natural language processing (NLP), machine learning and data mining. He has published three research papers in top NLP conferences.
\end{biography}
 % specify your biography file here (this example is biography.tex). 

\end{document}
